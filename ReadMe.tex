\documentclass[11pt,a4paper]{article}
\usepackage[utf8]{inputenc}
\usepackage[english]{babel}
\usepackage{amsmath}
\usepackage{amsfonts}
\usepackage{amssymb}
\usepackage{graphicx}
\usepackage[parfill]{parskip}
\author{Christoph Körner, 0726266, 932, ofice@chaosmail.at\\Marc Haubenstock 1525175, 932, marc.dhaubenstock@gmail.com}
\title{186.140 Real-time Rendering\\Read Me}
\begin{document}
\maketitle

\section{Read Me}

\subsection{Controls}

\begin{itemize}
\item Left Shift - Rotate Camera with Mouse
\item W,A,S,D - Forward,Left,Back,Right Respectively
\item R - Reset Camera
\item M - Toggle Wireframe
\end{itemize} 

\section{Technologies}

For the implementation of our project, we use \textit{C++}, \textit{OpenGL 4}, and additional helper tools such as \textit{Eigen} for linear algebra, \textit{OpenGP} for geometry processing, \textit{GLFW} for windows and inputs , \textit{GLEW} for extensions and \textit{SOIL} for textures. In addition, we are developing a Three.js-like framework for C++ called \textit{ThreeC++} in the branch \textsf{feat/threecpp}. We will use it in the second stage for abstractions of scene, camera, meshes, geometry and materials in order to allow easy composition of elements and effects in the scene.

\section{Effects}

\subsection{Skybox}

Our engine features a skybox with a texture taken from \newline
http://www.custommapmakers.org/skyboxes.php

\subsection{Billboarding}

Above the location of the camera there are two plane primitives with cloud textures.
These use a simple view oriented billboarding. 

\subsection{Water W.I.P}
To the bottom right of the camera spawn there is a tile with a simple reflection shader. This will be our water.

Of course all textures a subject to change.

\end{document}